\documentclass{tufte-book}

\usepackage{amsmath, amsthm}
\usepackage{graphicx}
\setkeys{Gin}{width=\linewidth,totalheight=\textheight,keepaspectratio}
\graphicspath{{graphics/}}

\title{STAT 244 \\ Homework 5}
\author{Joe Seidel}
\date{\today}

\usepackage{booktabs}
\usepackage{units}
\usepackage{fancyvrb}
\fvset{fontsize=\normalsize}
\usepackage{multicol}
\usepackage{lipsum}
\usepackage{pdfpages}
\usepackage{tikz}
\usepackage{wasysym}

\newcommand{\doccmd}[1]{\texttt{\textbackslash#1}}% command name -- adds backslash automatically
\newcommand{\docopt}[1]{\ensuremath{\langle}\textrm{\textit{#1}}\ensuremath{\rangle}}% optional command argument
\newcommand{\docarg}[1]{\textrm{\textit{#1}}}% (required) command argument
\newenvironment{docspec}{\begin{quote}\noindent}{\end{quote}}% command specification environment
\newcommand{\docenv}[1]{\textsf{#1}}% environment name
\newcommand{\docpkg}[1]{\texttt{#1}}% package name
\newcommand{\doccls}[1]{\texttt{#1}}% document class name
\newcommand{\docclsopt}[1]{\texttt{#1}}% document class option name
\DeclareMathOperator{\proj}{proj}
\newcommand{\vct}{\mathbf}


\newcommand{\dprod}[2]{\langle #1, #2 \rangle}
\newcommand{\Var}{\mathrm{Var}}
\newcommand{\Cov}{\mathrm{Cov}}
\newcommand{\MSE}{\mathrm{MSE}}
\newcommand{\MLE}{\mathrm{MLE}}

\newtheoremstyle{mytheoremstyle} % name
	{\topsep}		% Space above
	{\topsep}		% Space below
	{\itshape}		% Body font
	{}			% Indent amount
	{\bfseries}	% Theorem head font
	{\textnormal{:}}	% Punctuation after theorem head
	{.5em}		% Space after theorem head
	{}			%Theorem headspec
\theoremstyle{mytheoremstyle}
\newtheorem*{thm}{Thm.}

\newtheoremstyle{mylemstyle} % name
	{\topsep}		% Space above
	{\topsep}		% Space below
	{\itshape}		% Body font
	{}			% Indent amount
	{\bfseries}	% Theorem head font
	{\textnormal{:}}	% Punctuation after theorem head
	{.5em}		% Space after theorem head
	{}			%Theorem headspec
\theoremstyle{mylemstyle}
\newtheorem*{lem}{Lem.}


\newtheoremstyle{mydefstyle} % name
	{\topsep}		% Space above
	{\topsep}		% Space below
	{\normalfont}	% Body font
	{}			% Indent amount
	{\bfseries}	% Theorem head font
	{\textnormal{:}}	% Punctuation after theorem head
	{.5em}		% Space after theorem head
	{}			%Theorem headspec
\theoremstyle{mydefstyle}
\newtheorem*{mydef}{Def.}
\newtheorem*{ex}{E.g.}

\begin{document}

\maketitle
\pagenumbering{gobble}
\newpage
\pagenumbering{arabic}

\subsection{Question 1 Rice 8.16}

Consider an i.i.d. sample of random variables with density function
\[ f(x \mid \sigma) = \frac{1}{2\sigma}\exp\Big(- \frac{|x|}{\sigma} \Big) \]

\begin{enumerate}
\item Find the maximum likelihood estimate of $\sigma$.
\newthought{Suppose} we have i.i.d. random variables with result $x_1,x_2,..,x_n$.  Then

\begin{align*}
f(x_1, x_2,...,x_n\mid \sigma) &= f(x_1\mid \sigma) \cdot f(x_2\mid \sigma) \cdot \cdot \cdot f(x_n\mid \sigma)\\
&= \frac{1}{2\sigma}e^{-\frac{|x_1|}{\sigma}} \cdot \frac{1}{2\sigma}e^{-\frac{|x_2|}{\sigma}} \cdot \cdot \cdot \frac{1}{2\sigma}e^{-\frac{|x_n|}{\sigma}}\\
&= \prod_{i=1}^{n}\frac{1}{2\sigma}e^{-\frac{|x_i|}{\sigma}}\\
\end{align*}

So the likelihood function of $\sigma$ is given

\[L(\sigma) = \prod_{i=1}^{n}\frac{1}{2\sigma}e^{-\frac{|x_i|}{\sigma}}. \]

Estimate $\hat{\sigma}$ by taking the log and differentiate then solve by setting to $0$.

\begin{align*}
\frac{d}{d\sigma}\log(L(\sigma)) &= \frac{d}{d\sigma} -n\log(2)-n\log(\sigma) - \sum_{i=1}^n \frac{|x_i|}{\sigma} \\
&= -\frac{n}{\sigma} + \frac{\sum_{i=1}^n|x_i|}{\sigma^2}\\
\end{align*}

Setting the above equal to $0$ gives the result \marginnote{In part 2 of this question, we'll verify that that this maximizes}

\[ \hat{\sigma} = \frac{1}{n} \sum_{i=1}^n |x_i| \]

\item Find the asymptotic variance of the mle.

\[
\frac{1}{\tau^2} = -E\Big[\frac{d^2}{d\sigma^2}\log L(\sigma)\Big]
\]

\begin{align*}
\frac{d^2}{d\sigma^2}\log(L(\sigma)) &= \frac{n}{\sigma^2} - \frac{2\sum_{i=1}^n |x_i|}{\sigma^3}\\
&= \frac{n}{\hat{\sigma}^2} - \frac{2n}{\hat{\sigma}^2}\\
&= -\frac{n}{\hat{\sigma}^2}\\
\end{align*}
 \marginnote{We sub $\sigma$ with the estimator to elimate the summation term.}

Plugging the result into the equation that began this section

\begin{align*}
\frac{1}{\tau^2} &= -E[-\frac{n}{\hat{\sigma}^2}]\\
&= \frac{n}{\hat{\sigma}^2}
\end{align*}

Finally, $\tau^2 = \frac{\hat{\sigma}^2}{n}$.

\end{enumerate}


\subsection{Question 2}
Suppose that $X_1, X_2, ..., X_n$ are i.i.d. random variables on the interval $[0,1]$ with the density function

\[f(x\mid \alpha) = \frac{\Gamma(3\alpha)}{\Gamma(\alpha)\Gamma(2\alpha)}x^{\alpha-1}(1-x)^{2\alpha -1} \]
where $\alpha > 0$ is a parameter to be estimated from the sample.  It can be shown that
\[ E(X) = \frac{1}{3} \]
\[ \Var(X) = \frac{2}{9(3\alpha+1)}\]

\begin{enumerate}
\item What equation does the mle of $\alpha$ satisfy.

\begin{align*}
L(\alpha) &= f(x_1 \mid \alpha)\cdot f(x_2 \mid \alpha) \cdot\cdot\cdot f(x_n \mid \alpha)\\
&= \prod_{i=1}^n \frac{\Gamma(3\alpha)}{\Gamma(\alpha)\Gamma(2\alpha)} x_i^{\alpha-1}(1-x_i)^{2\alpha -1}\\
\end{align*}
\begin{align*}
\log(L(\alpha)) &= n\log(\Gamma(3\alpha))-n\log(\Gamma(\alpha))-n\log(\Gamma(2\alpha))\\
 & + \sum_{i=1}^n[(\alpha-1)\log(x_i) + (2\alpha-1)\log(1-x_i)]\\
\end{align*}
\begin{align*}
\frac{d}{d\alpha}\log(L(\alpha)) &=\frac{n3 \Gamma'(3\alpha)}{\Gamma(3\alpha)} -\frac{n \Gamma'(\alpha)}{\Gamma(\alpha)} - \frac{n2 \Gamma'(2\alpha)}{\Gamma(2\alpha)}\\
&+  \sum_{i=1}^n[\log(x_i) + 2\log(1-x_i)]\\
\end{align*}
Setting the above equal $0$.
\[
\frac{3 \Gamma'(3\alpha)}{\Gamma(3\alpha)} -\frac{ \Gamma'(\alpha)}{\Gamma(\alpha)} - \frac{2 \Gamma'(2\alpha)}{\Gamma(2\alpha)}
= -\frac{1}{n} \sum_{i=1}^n[\log(x_i) + 2\log(1-x_i)]
\]

\item What is the asymptotic variance of the mle.

Use the property
\[ \frac{1}{\tau^2} = -E\Big[\frac{d^2}{d\alpha^2}\log(L(\alpha))\Big] \]
\begin{align*}
\frac{d^2}{d\alpha^2} &= \frac{n9 \Gamma''(3\alpha)}{\Gamma(3\alpha)} - \frac{n9 \Gamma'(3\alpha)^2}{\Gamma(3\alpha)^2} -\frac{n \Gamma''(\alpha)}{\Gamma(\alpha)} +\frac{n \Gamma'(\alpha)^2}{\Gamma(\alpha)^2}\\
 &- \frac{n4 \Gamma''(2\alpha)}{\Gamma(2\alpha)} + \frac{n4 \Gamma'(2\alpha)^2}{\Gamma(2\alpha)^2} \\
\end{align*}
\begin{align*}
\frac{1}{\tau^2} &= -E\Big[n(\frac{9 \Gamma''(3\alpha)}{\Gamma(3\alpha)} - \frac{9 \Gamma'(3\alpha)^2}{\Gamma(3\alpha)^2} -\frac{ \Gamma''(\alpha)}{\Gamma(\alpha)} +\frac{ \Gamma'(\alpha)^2}{\Gamma(\alpha)^2}\\
 &- \frac{4 \Gamma''(2\alpha)}{\Gamma(2\alpha)} + \frac{4 \Gamma'(2\alpha)^2}{\Gamma(2\alpha)^2}\Big] \\
\end{align*}
Hence
\begin{multline}
\tau^2 = -\frac{1}{n}\big[\frac{9 \Gamma''(3\alpha)}{\Gamma(3\alpha)} - \frac{9 \Gamma'(3\alpha)^2}{\Gamma(3\alpha)^2} -\frac{ \Gamma''(\alpha)}{\Gamma(\alpha)} +\frac{ \Gamma'(\alpha)^2}{\Gamma(\alpha)^2}\\
 - \frac{4 \Gamma''(2\alpha)}{\Gamma(2\alpha)} + \frac{4 \Gamma'(2\alpha)^2}{\Gamma(2\alpha)^2}\Big]^{-1}
\end{multline}
\end{enumerate}

\subsection{Question 3}
Suppose that $X$ is the number of successes in a Binomial experiment with $n$ trials and probability of success $\theta/(1+\theta)$, where $0\leq \theta < \infty$.

\begin{enumerate}
\item Find the MLE of $\theta$.

\newthought{Consider}
\[p(x\mid \theta) = \binom{n}{X}\Big(\frac{\theta}{1+\theta}\Big)^X\big(1-\frac{\theta}{1+\theta} \Big)^{n-X}. \]

Then
\[L(\theta) = \binom{n}{X}\Big(\frac{\theta}{1+\theta}\Big)^X\big(1-\frac{\theta}{1+\theta} \Big)^{n-X}. \]

The log likelihood is
\begin{align*}
\log L(\theta) &= \log\binom{n}{X} + X\log(\theta) - X\log(1+\theta) - (n-X)\log(1+\theta)\\
&= \log\binom{n}{X} + X\log(\theta) - n\log(1+\theta) \\
\end{align*}
Differentiating,
\begin{align*}
\frac{d}{d\theta} \log L(\theta) &= \frac{X}{\theta} - \frac{n}{\theta+1}\\
\end{align*}

Setting the above to $0$ results in
\[ \hat{\theta} = \frac{X}{n-X} \]
furthermore evaluating the second derivative using $\hat{\theta}$
\begin{align*}
\frac{d^2 \log L(\theta)}{d\theta^2} &= -\frac{X}{\theta^2} + \frac{n}{(\theta+1)^2}\\
&= -\frac{X}{\hat{\theta}^2} + \frac{n}{(\hat{\theta}+1)^2}\\
&= -\frac{(n - X)^2}{X} + \frac{(n-X)^2}{n} < 0\\
\end{align*}
verifying that this maximizes $\hat{\theta}$.

\item Use Fisher's Theorem to find the approximate distribution of the MLE when $n$ is large.

Fisher's approximation allows the result for large $n$ in the given case, $\hat{\theta}$ will have approximately a $N(\theta, \frac{\tau^2(\theta)}{n})$ distribution.

It remains to find $\tau^2(\theta)$.

\[ \frac{1}{\tau^2(\theta)} = -E\Big[ \frac{d^2}{d\theta^2} \log f(X\mid \theta)\Big]. \]
\begin{align*}
E[\frac{X}{\theta^2} + \frac{n}{(\theta+1)^2}] &= -\frac{E(X)}{\theta^2} + \frac{n}{(\theta+1)^2}\\
&= -\frac{n}{(1+\theta)\theta} + \frac{n}{(1+\theta)^2}\\
&= -\frac{n}{(1+\theta)^2\theta}\\
\end{align*}

Hence $\tau^2(\theta) = \frac{\theta(1+\theta)^2}{n}$

Applying Fisher, $\hat{\theta}$ \marginnote{It is important to recognize that we evaulated the entire likelihood function, so we take $\tau^2(\theta)$ as is.}will have an approximately normal $N(\theta,\frac{\theta(1+\theta)^2}{n})$ distribution.

\end{enumerate}

\subsection{Question 4}
If $X$ and $Y$ are independent each with a Poisson distribtuion, show that $Z=X+Y$ is Poisson distributed.

\newthought{Since} $X$ and $Y$ are independent, use the discrete convolution formula.  I'll assume $X$ and $Y$ have paramters $\lambda_1$ and $\lambda_2$ respectively.  However, the result will be obvious if the parameter $\lambda$ is the same.

\[ P(X+Y=z) = p_Z(z) = \sum_{x=0}^z p_X(x) p_Y(z-x) \]
\begin{align*}
p_Z(z) &= \sum_{x=0}^z e^{-\lambda_1} \frac{\lambda_1^x}{x!} e^{-\lambda_2} \frac{\lambda_2^{z-x}}{(z-x)!}\\
&= e^{-(\lambda_1 + \lambda_2)} \sum_{x=1}^z \frac{\lambda_1^x}{x!} \frac{\lambda_2^{z-x}}{(z-x)!}\\
&= e^{-(\lambda_1 + \lambda_2)}\frac{1}{z!} \sum_{x=1}^z \frac{z!}{x!(z-x)!} \lambda_1^x \lambda_2^{z-x}\\
&= e^{-(\lambda_1 + \lambda_2)}\frac{1}{z!} (\lambda_1+\lambda_2)^z\\
&= e^{-(\lambda_1 + \lambda_2)}\frac{(\lambda_1+\lambda_2)^z}{z!}\\
\end{align*}

Which is a Poisson probability distribution.

\subsection{Question 5}
In a famous example, Bortkiewicz tabuluated the number of Cavalry men kicked to death by horses in the Prussian Cavalry, for 14 Corps over 20 years (1875-1894), giving $n=280$ observations in all.

\begin{table}
\centering
\caption{Frequency tabulation}
\label{freq-tab}
\begin{tabular}{|l|l|lll}
\cline{1-2}
Number of deaths & Frequency count \\ \cline{1-2}
0                & 144             \\ \cline{1-2}
1                & 91              \\ \cline{1-2}
2                & 32              \\ \cline{1-2}
3                & 11              \\ \cline{1-2}
4                & 2               \\ \cline{1-2}
More             & 0               \\ \cline{1-2}
Total            & 280             \\ \cline{1-2}
\end{tabular}
\end{table}

\begin{enumerate}

\item For this model, find the MLE of $\theta$, assume the $X_i$'s are independent.

The likelihood function of $\theta$ is

\[ L(\theta) = e^{-\theta n}  \frac{\theta^{\sum_{i=1}^n x_i}}{ \prod_{i=1}^n \frac{1}{x_i!}}. \]

Differentiating the log of this function gives

\[ \frac{d}{d\theta}\log L(\theta) = \frac{\sum_{i=1}^n x_i}{\theta} - n. \]

Setting the above equal to zero we have

\[ \hat{\theta} = \frac{1}{n} \sum_{i=1}^n x_i = \overline{X} \]

\item Find the MSE of the MLE

\newthought{Since} the MLE is the sample mean the MLE is unbiased.

\[\MSE = \Var(\hat{\theta}) = \Var(\overline{X}) =  \frac{\Var(X_i)}{n} \]

Since $P(X_i=k\mid \theta)$ is Poisson distributed, $\Var(X_i)=\theta$.

So $\MSE = \frac{\theta}{n}$.

\item From the Central Limit Theorem find the approximate distribution of the MLE when $n$ is large.

Since the MLE is the sample mean for i.i.d $X_i$ where $E(X_i)=\theta$ and $\Var(X_i)=\theta$, $\hat{\theta}$ has an approximately $N(\theta,\frac{\theta}{n})$ distrubution, which becomes a better aproximiation as $n$ gets larger.

\item From Fisher's Approximation, find the approximate distribution of the MLE when $n$ is large.

\newthought{Fisher's} approximation states that if data consit of independent random variables each with distribution $f(x\mid\theta)$, and $\hat{\theta}$ can be found by solving log likelihood equal to 0, then for large $n$, $\hat{\theta}$ has approximately a $N(\theta, \frac{\tau^2(\theta)}{n})$.

\[ \frac{1}{\tau^2(\theta)} = -E\Big[ \frac{d^2}{d\theta^2}\log p(X_1\mid \theta)\Big] \]
Where
\[ p(X_1 \mid \theta) = e^{-\theta} \frac{\theta^x_1}{x_1!}. \]

Taking the log of the above yields
\[\log p(X_1 \mid \theta) = -\theta + x \log(\theta) - \log(x!). \]

Differientiating,
\[ \frac{d}{d\theta}\log p(X_1 \mid \theta) = \frac{x_1}{t}-1 \]
\[ \frac{d^2}{d\theta^2}\log p(X_1 \mid \theta) = -\frac{x_1}{\theta^2} \]

Plugging this into the formula for asymptotic variance,

\[ \frac{1}{\tau^2} = -E\Big[-\frac{x_1}{\theta^2}\Big] = \frac{E[X_1]}{\theta^2} \]

Recall that $E(X_1) = \theta$ so $\tau^2(\theta) = \theta$.  Therefore, from Fisher the distrubition of $\hat{\theta}$ can be approximated as $N(\theta, \frac{\theta}{n})$.

\item Evaluate the MLE for the given data.

We found MLE to be the sample mean.  So take the total number of deaths by horse kicks and divide by $n=280$.
\[ \MLE = \hat{\theta} = \frac{196}{280}= .7 \]

\item If $\theta$, the mean number of deaths per Corps in a year, is really 1.0, what (approximately) is the probability that the MLE would turn out to be below $.85$.

\newthought{Since} $\hat{\theta}$ has approximately $N(\theta, \frac{\theta}{n})$ distribution calculate this using

\[P(\hat{\theta} < .85) = \Phi\Big(\frac{.85-1}{\sqrt{\frac{1}{280}}}\Big) = .0062\]
\end{enumerate}

\subsection{Question 6}
Let $X_1,...,X_n$ be an i.i.d. sample from a Rayleigh distributions with parameter $\theta>0$:
\[ f(x\mid \theta) = \frac{x}{\theta^2}e^{-x^2/(2\theta^2)} \text{ , } x \geq 0 \]

\begin{enumerate}
\item Find the mle of $\theta$
\newthought{The} likelihood function of of $\theta$ is
\begin{align*}
L(\theta) &= f(x_1,...,x_n \mid \theta)\\
&= f(x_1 \mid \theta) \cdot \cdot \cdot f(x_n \mid \theta)\\
&= \prod_{i=1}^n \frac{x_i}{\theta^2}e^{-x_i^2/2\theta^2}\\
\end{align*}

Taking the log gives

\[ \log L(\theta) = \sum_{i=1}^n \Big[ \log(x_i) - 2\log(\theta) - \frac{x_i^2}{2\theta^2} \Big]. \]

Differentiating
\[ \frac{d}{d\theta} \log L(\theta) = -\frac{2n}{\theta} + \frac{\sum_{i=1}^n x_i^2}{\theta^3}. \]

Setting the derivative of the log likelihood to zero yields

\[\hat{\theta} = \sqrt{\frac{\sum_{i=1}^n x_i^2}{2n}} \]
\item Find the asymptotic variance of the mle.

\begin{align*}
\frac{d^2}{d\theta^2} \log L(\theta)  &= \frac{d^2}{d\theta^2} -\frac{2n}{\theta} + \frac{\sum_{i=1}^n x_i^2}{\theta^3}\\
&= \frac{2n}{\theta^2} - \frac{3 \sum_{i=1}^n x_i^2}{\theta^4} \\
&= \frac{2n}{\hat{\theta}^2} - \frac{6n}{\hat{\theta}^2}\\
&= -\frac{4n}{\hat{\theta}^2}\\
\end{align*}

\[ \frac{1}{\tau^2(\theta)} = -E \Big[ -\frac{4n}{\theta^2}\Big] \]

So the asymptotic variance is $\frac{\theta^2}{4n}$
\end{enumerate}

\end{document}
\grid
