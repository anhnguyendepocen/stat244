\documentclass{tufte-book}

\usepackage{amsmath, amsthm}
\usepackage{graphicx}
\setkeys{Gin}{width=\linewidth,totalheight=\textheight,keepaspectratio}
\graphicspath{{graphics/}}

\title{STATS 244 \\ Homework 2}
\author{Joe Seidel}
\date{\today}

\usepackage{booktabs}
\usepackage{units}
\usepackage{fancyvrb}
\fvset{fontsize=\normalsize}
\usepackage{multicol}
\usepackage{lipsum}
\usepackage{pdfpages}
\usepackage{tikz}
\usepackage{wasysym}

\newcommand{\doccmd}[1]{\texttt{\textbackslash#1}}% command name -- adds backslash automatically
\newcommand{\docopt}[1]{\ensuremath{\langle}\textrm{\textit{#1}}\ensuremath{\rangle}}% optional command argument
\newcommand{\docarg}[1]{\textrm{\textit{#1}}}% (required) command argument
\newenvironment{docspec}{\begin{quote}\noindent}{\end{quote}}% command specification environment
\newcommand{\docenv}[1]{\textsf{#1}}% environment name
\newcommand{\docpkg}[1]{\texttt{#1}}% package name
\newcommand{\doccls}[1]{\texttt{#1}}% document class name
\newcommand{\docclsopt}[1]{\texttt{#1}}% document class option name
\DeclareMathOperator{\proj}{proj}
\newcommand{\vct}{\mathbf}


\newcommand{\dprod}[2]{\langle #1, #2 \rangle}
\newcommand{\Var}{\mathrm{Var}}
\newcommand{\Cov}{\mathrm{Cov}}

\newtheoremstyle{mytheoremstyle} % name
	{\topsep}		% Space above
	{\topsep}		% Space below
	{\itshape}		% Body font
	{}			% Indent amount
	{\bfseries}	% Theorem head font
	{\textnormal{:}}	% Punctuation after theorem head
	{.5em}		% Space after theorem head
	{}			%Theorem headspec
\theoremstyle{mytheoremstyle}
\newtheorem*{thm}{Thm.}

\newtheoremstyle{mylemstyle} % name
	{\topsep}		% Space above
	{\topsep}		% Space below
	{\itshape}		% Body font
	{}			% Indent amount
	{\bfseries}	% Theorem head font
	{\textnormal{:}}	% Punctuation after theorem head
	{.5em}		% Space after theorem head
	{}			%Theorem headspec
\theoremstyle{mylemstyle}
\newtheorem*{lem}{Lem.}


\newtheoremstyle{mydefstyle} % name
	{\topsep}		% Space above
	{\topsep}		% Space below
	{\normalfont}	% Body font
	{}			% Indent amount
	{\bfseries}	% Theorem head font
	{\textnormal{:}}	% Punctuation after theorem head
	{.5em}		% Space after theorem head
	{}			%Theorem headspec
\theoremstyle{mydefstyle}
\newtheorem*{mydef}{Def.}
\newtheorem*{ex}{E.g.}

\begin{document}

\maketitle
\pagenumbering{gobble}
\newpage
\pagenumbering{arabic}

\subsection{Rice Chapter 2, 44}
Let $T$ be an exponentional random variable with parameter $\lambda$.   Let $X$ be a discrete random variable defined as $X=k$ if $k \leq T <k+1$, $k=0,1,...$. Find the frequency function of $X$.

\newthought{The exponential} density function is
\[ f(x)=
\begin{cases}
      \lambda e^{-\lambda x},  & \text{if}\ x \geq 0\\
      0, & \text{if}\ x<0\\
\end{cases}
\]
The random variable $X$ is essentially the space between two consectutive integers on $f(x)$, $k$ and $k+1$.  Therefore, using the CDF of $f(x)$, we can find the frequency function of $X$.
\[F_T(x)=\int_{0}^{x}f(u)du=1-e^{-\lambda x}\]
\begin{align*}
P(k\leq T<k+1) &= P(T \leq k+1) - P(T<k)\\
&= F_T(k+1) - F_T(k)\\
&= (1-e^{-\lambda(k+1)}) - (1-e^{-\lambda(k)})\\
&= -e^{-\lambda k- \lambda}+e^{-\lambda k}\\
&= -e^{-\lambda k}(e^{-\lambda}-1)
\end{align*}

So the frequency function of $X$ is $f_X(x)=  -e^{-\lambda x}(e^{-\lambda}-1)$


\subsection{Rice Chapter 2, 46}
$T$ is an exponential random variable, and $P(T<1) = .05$.  What is $\lambda$?

\newthought{Using the CDF} of an exponential random variable, we find...
\begin{align*}
p(T<1)&=.05\\
1-e^(-\lambda) &= .05\\
-e^(-\lambda) &= -.95\\
e^(-\lambda) &= \frac{19}{20}\\
e^{\lambda} &= \frac{20}{19}\\
\lambda &= \ln(\frac{20}{19})\\
\lambda &\approx .051
\end{align*}


\subsection{Rice Chapter 3, 8}
Let $X$ and $Y$ have the joint density
\[f(x,y) = \frac{6}{7}(x+y)^2 ,\  0\leq x\leq1 ,\ 0\leq y\leq 1\]

\begin{enumerate}
\item By integrating over the appropriate regions, find (i) $P(X>Y)$, (ii) $P(X+Y \leq 1)$, (iii) $P(X \leq \frac{1}{2})$.]\

\begin{enumerate}
\item $P(X>Y)$
\[ P(X>Y) = \int_{-\infty}^{\infty}\int_{-\infty}^{x}f_{XY}dydx \]
Since the joint density function applies to $0\leq x \leq 1$ set the limit of integration for $X$ to be $0,1$ and $0,x$ for $Y$.
\begin{align*}
P(X>Y) &= \int_{0}^{1}\int_{0}^{x}\frac{6}{7}(x+y)^2dydx\\
&= \int_{0}^{1}\frac{6}{7}[\frac{1}{3}(x+y)^3\Big|_0^x]dx\\
&= \int_{0}^{1}\frac{6}{7}[\frac{1}{3}(2x)^3 - \frac{1}{3}(x)^3]dx\\
&= \int_{0}^{1}\frac{6}{7}[\frac{1}{3}7x^3]dx\\
&=\int_{0}^{1}2x^3dx\\
&=\frac{1}{2}x^4\Big|_0^1\\
&= \frac{1}{2}
\end{align*}

\item $P(X+Y\leq 1)$
\begin{align*}
P(X+Y \leq 1) &= \int_{0}^{1}\int_{0}^{1-x}\frac{6}{7}(x+y)^2dydx\\
&= \int_{0}^{1}\frac{6}{7}[\frac{1}{3}(x+y)^3\Big|_0^{1-x}]dx\\
&=\int_{0}^{1}\frac{6}{7}[\frac{1}{3}(x+1-x)^3-\frac{1}{3}(x+0)^3]dx\\
&=\int_{0}^{1}\frac{6}{7}[\frac{1}{3}(1-x^3)]dx\\
&=\frac{2}{7}[(x-\frac{1}{4}x^4)\Big|_0^1]\\
&= \frac{3}{14}\\
\end{align*}

\item $P(X \leq \frac{1}{2})$
\begin{align*}
P(X\leq \frac{1}{2}) &=  \int_{0}^{\frac{1}{2}}\int_{0}^{1}\frac{6}{7}(x+y)^2dydx\\
&= \int_{0}^{\frac{1}{2}}\frac{6}{7}[\frac{1}{3}(x+y)^3\Big|_0^1]dx\\
&=\int_{0}^{\frac{1}{2}}\frac{6}{7}[\frac{1}{3}((x+1)^3-x^3)]dx\\
&=\int_{0}^{\frac{1}{2}}\frac{2}{7}[((x+1)^3-x^3)]dx\\
&= \frac{2}{7}[\frac{1}{4}(x+1)^4-\frac{1}{4}x^4\Big|_0^\frac{1}{2}]\\
&=\frac{2}{28}[(\frac{3}{2}^4 - \frac{1}{2}^4) - 1]\\
&=\frac{2}{14}
\end{align*}
\end{enumerate}

\item Find the marginal densities of $X$ and $Y$.
\begin{align*}
f(y) &= \int_{-\infty}^\infty f(x,y)dx\\
&=\int_{0}^1 \frac{6}{7}(x+y)^2dx\\
&=\frac{6}{7}[\frac{1}{3}(x+y)^3\Big|_0^1]\\
&=\frac{2}{7}(1+y)^3+y^3\\
&=\frac{2}{7}(3y^2+3y+1) ,\ 0 \leq y \leq 1
\end{align*}

\begin{align*}
f(x) &= \int_{-\infty}^\infty f(x,y)dy\\
&=\int_{0}^1 \frac{6}{7}(x+y)^2dy\\
&=\frac{6}{7}[\frac{1}{3}(x+y)^3\Big|_0^1]\\
&=\frac{2}{7}(1+x)^3+x^3\\
&=\frac{2}{7}(3x^2+3x+1) ,\ 0 \leq x \leq 1
\end{align*}

\item Find the two conditional densities.

\begin{align*}
f(x\mid y) &= \frac{f(x,y)}{f(y)}\\
&=\frac{\frac{6}{7}(x+y)^2}{\frac{2}{7}(3y^2+3y+1)}\\
&=\frac{3(x+y)^2}{3y^2+3y+1}\\
\end{align*}

\begin{align*}
f(y\mid x) &= \frac{f(x,y)}{f(x)}\\
&=\frac{\frac{6}{7}(x+y)^2}{\frac{2}{7}(3x^2+3x+1)}\\
&=\frac{3(x+y)^2}{3x^2+3x+1}\\
\end{align*}
\end{enumerate}

\subsection{Rice Chapter 3, 14}

Suppose that
\[f(x,y)=xe^{-x(y+1)}, ,\ 0\leq x <\infty ,\ 0\leq y <\infty \]

\begin{enumerate}

\item Find the marginal density of $X$ and $Y$.  Are $X$ and $Y$ independent?

\marginnote{Switch terms $u = x(-y)-x$ so $du=-xdy$}
\begin{align*}
f_X(x) &= \int_0^\infty f_{XY}dy\\
&= \int_0^\infty xe^{-x(y+1)}dy\\
&= x\int_0^\infty e^{x(-y)-x}dy\\
&= -\int_0^\infty e^udu\\
&= -e^u\Big|_0^\infty\\
&= -e^{-x(y+1)}\Big|_0^\infty\\
&= 0-(-e^{-x})\\
\end{align*}
So $f_X(x) = e^{-x}$.

To find
\[ f_Y(y) = \int_0^\infty xe^{-x(y+1)} \]
We'll integrate by parts, $\int fdg = fg -\int gdf$ where
\begin{align*}
f &=x\\
df &=dx\\
dg &= e^{x(-y-1)}dx\\
g &=\frac{e^{x(-y-1)}}{-y-1}
\end{align*}

\begin{align*}
f_Y(y)&=\int_0^\infty xe^{-x(y+1)}dx\\
&= \frac{xe^{x(-y-1)}}{-y-1}\Big|_0^\infty-\frac{1}{-y-1}\int_0^\infty e^{x(-y-1)}dx\\
\end{align*}

Substitute terms $u=x(-y-1)$ and $du=(-y-1)dx$.
\begin{align*}
f_Y(y)&= 0-\frac{1}{(-y-1)^2}\int_0^\infty e^{u}du\\
&=-\frac{1}{(-y-1)^2}[e^{u}\Big|_0^\infty]\\
&=\frac{1}{(y+1)^2}
\end{align*}

Therefore, $f_Y(y)=\frac{1}{(y+1)^2}$.

We say $X,Y$ are independent if $f(x,y)=f(x)f(y)$ but
\[ xe^{-x(y+1)}\neq \frac{e^{-x}}{(y+1)^2}. \]
So $X$ and $Y$ are not independent.

\item Find the conditional densities of $X$ and $Y$.
\begin{align*}
f(x\mid y) &= \frac{f(x,y)}{f(y)}\\
&= \frac{xe^{-x(y+1)}}{\frac{1}{(y+1)^2}}\\
&= (y+1)^2xe^{-x(y+1)}\\
\end{align*}

\begin{align*}
f(x\mid y) &= \frac{f(x,y)}{f(x)}\\
&= \frac{xe^{-x(y+1)}}{e^{-x}}\\
&= xe^{-xy}
\end{align*}
\end{enumerate}


\subsection{Question 5}
The Pareto distributions are a family of distributions of a continuous random variable $X$ with probability density function given by
\[f(x;\alpha,\theta)=
\begin{cases}
      \frac{\alpha\theta^{\alpha}}{x^{\alpha+1}}  & \text{for}\ x \geq \theta\\
      0 & \text{for}\ x<\theta\\
\end{cases}
\]

where $\alpha > 0$ and $\theta > 0$ are the paremeters of the family.  The Pareto distribution arises as a model for the distribution of sizes  of some measured quantity, such as personal or corperate income, or city population, or size of firm, given that it exceeds the threshold $\theta$.

\begin{enumerate}
\item Verify that this the formula for a density.

\newthought{When} $x > \theta$ we have $f(x)= \frac{\alpha\theta^{\alpha}}{x^{\alpha+1}}$.  To show that this is a density, the integral must sum to $1$.

\begin{align*}
\int_{-\infty}^\infty f_Xdx &= \int_{-\infty}^\infty \frac{\alpha\theta^{\alpha}}{x^{\alpha+1}}dx\\
&= \int_\theta^\infty \alpha\theta^{\alpha}x^{-(\alpha+1)}dx\\
&= \alpha\theta^\alpha[ \frac{-x^{-\alpha}}{\alpha}\Big|_\theta^\infty]\\
&= \alpha\theta^\alpha[0-(\frac{-1}{\alpha\theta^{\alpha}}]\\
&= 1
\end{align*}

\item Find the formula for the cumulative distribution function.

\begin{align*}
F(x) &= \int_{-\infty}^\infty f(x)dx\\
&= \int_\theta^x \frac{\alpha\theta^{\alpha}}{x^{(\alpha+1)}}dx\\
&= \alpha\theta^\alpha \int_\theta^x x^{-(\alpha+1)}dx\\
&= \alpha\theta^\alpha [\frac{-1}{\alpha x^\alpha}\Big|_\theta^x]\\
&= 1 - (\frac{\theta}{x})^\alpha\\
\end{align*}
\item Assuming $\alpha > 1$, find $E(X)$.  What is $E(X)$ if $0 < \alpha \leq 1$?

\begin{align*}
E(X) &= \int_{-\infty}^{\infty} xf_X(x)dx\\
&= \int_\theta^\infty x\frac{\alpha\theta^{\alpha}}{x^{\alpha+1}}dx\\
&= \alpha\theta^\alpha \int_\theta^x x^{-\alpha}dx\\
&= \alpha\theta^\alpha [\frac{x^{1-\alpha}}{1-\alpha}\Big|_\theta^\infty]\\
&= \alpha\theta^\alpha [ \frac{-1}{(\alpha-1)(x^{\alpha-1})}\Big|_\theta^\infty]\\
&= \alpha\theta^\alpha[0-\frac{-1}{(\alpha-1)(\theta^{\alpha-1})}]\\
&= \frac{\alpha\theta}{\alpha-1} ,\ \alpha > 1\\
\end{align*}

When $0 <\alpha \leq 1$ the integral does not converge.  Therefore
\[ E(X)=
\begin{cases}
    \frac{\alpha\theta}{\alpha-1}  & \text{for}\ \alpha > 1\\
    \text{DNE} & \text{for}\ 0<\alpha \leq1\\
\end{cases}
\]

\item Show that if $\alpha >2$, $\Var(X) = \frac{\alpha\theta^2}{(\alpha-1)^2(\alpha-2)}$


\[  = E(X^2)-\mu_x^2 \]

Since $\alpha>1$ we already know $E(X)$.  To find $E(X^2)$.
\begin{align*}
E(X^2) &= \int_\theta^\infty x^2 \frac{\alpha\theta^{\alpha}}{x^{\alpha+1}}dx\\
&=\int_\theta^\infty \frac{\alpha\theta^{\alpha}}{x^{\alpha-1}}dx\\
&= \alpha\theta^{\alpha}[\frac{-1}{(\alpha-2)x^{\alpha-2}}\Big|_\theta^\infty] ,\ \alpha > 2\\
&=\alpha\theta^{\alpha}[\frac{1}{(\alpha-2)(\theta^{\alpha-2})}]\\
&= \frac{\alpha\theta^2}{\alpha-2} ,\ \alpha >2\\
\end{align*}

Plugging what we just found back into the equation for variance, we have...
\[ \Var(X) = \frac{\alpha\theta^2}{\alpha-2} - \frac{\alpha\theta}{\alpha-1}  \]
So $\Var(X) = \frac{\alpha\theta^2}{(\alpha-2)(\alpha-1)^2}$ when $\alpha >2$.

\item Find (in terms of $\alpha$ and $\theta$) the median of the distribution of $X$.

\newthought{The median} can be found using the CDF.
\begin{align*}
\frac{1}{2} &= 1-(\frac{\theta}{x})^\alpha\\
\frac{1}{2}&=(\frac{\theta}{x})^\alpha\\
x^\alpha &= \theta^\alpha\\
\end{align*}

So $\tilde{\mu} = \theta^\alpha\sqrt{2}$

\item Suppose $X$ has a Pareto distribution with $\alpha=3$ and $\theta=1$.  Find $P(1<X<4)$ and $P(4<X<5)$.

\newthought{Since} $P(1<X<4) = P(X<4)-P(X<1)$ we consider the CDF.

\begin{align*}
P(X<4)-P(X<1) &= 1-(\frac{1}{4})^3 - (1-(\frac{1}{1})^3)\\
&=  \frac{63}{64}
\end{align*}

Similarily, $(P(4<X<5)=\frac{61}{8000}$
\end{enumerate}


\subsection{Question 6}
Suppose $X$ and $Y$ are independent random variables with $X$ following a uniform distribution on $(0,1)$ and $Y$ exponentially distributed with parameter $\lambda=1$.

\begin{enumerate}

\item Find the density for $Z=X+Y$ (be careful with your limits of integration).  Sketch the density and verify directly that it integrates to 1.  Find the medain (the value of $Z$ for which $P(Z\leq z) = \frac{1}{2}$.

\newthought{The functions} PDFs can be writtin...
\[ f_X = 1 \]
and
\[ f_Y=e^{-y}.\]

Since $X$ and $Y$ are independent we can find the density of using
\[ f_Z(z) = \int_{-\infty}^{\infty}f_X(z-x)f_Y(y)dy \]
However, there are two cases to consider since.
\[ f_x(z-y)=
\begin{cases}
        1  & \text{for}\ 0 \leq z-y \leq 1\\
        0 & \ \text{otherwise}\\
\end{cases}
\]

When $0 \leq z \leq 1$ and $z>1$.   For the first we have.
\begin{align*}
f_Z(z) &= \int_0^z e^{-y}(1)dy\\
&= 1-e^{-z} ,\ 0\leq z \leq 1\\
\end{align*}

When $z>1$, we can say that $z=y$ and we have
\begin{align*}
f_Z(z) &= \int_{z-1}^z e^{-y}(1)dy\\
&= e^{-(z-1)}-e^{-z}\\
\end{align*}

Therefore the PDF can be written
\[f_Z(z) =
\begin{cases}
    1-e^{-z} & 0 \leq z \leq 1\\
    e^{-(z-1)}-e^{-z} & z>1\\
\end{cases}
\]

\newthought{To verify} this is a formula for a density we take (let $u=-z$ and $s=1-z$.
\begin{align*}
\int_0^1  1-e^{-z}  &+ \int_{1}^\infty e^{-(z-1)}-e^{-z}\\
-\int_0^1 e^{-z} +\int_0^1 1dz &+ -\int_1^\infty e^{-z}dz + \int_1^\infty e^{1-z}\\
\int_0^1 e^udu + z &+ \int_1^\infty e^{u}du - \int_1^\infty e^{s}ds \\
[z+ e^{-z}\Big|_0^1] &+ [-(e-1)e^{-z}\Big|_1^\infty]\\
\end{align*}
Which equals $1$, verifying an equation for density!

\newthought{Next} we need to find the median.  The familiar way to do this, starts with the CDF.

\begin{align*}
F_z(z) &= \int_0^1 \int_0^{z-x}f(x,y)dydx\\
\text{let $y = v-x$}&\\
&= \int_0^1 \int_0^{z}f(x,v-x)dvdx\\
&= \int_0^1 \int_0^{z}(1)e^{-(v-x)}dvdx\\
\text{let $u=x-v$}& \text{ and $du=-dv$}&\\
&= \int_0^1-\int_x^{z-x}e^ududx\\
&= \int_0^1[-e^u\Big|_x^{x-z}]dx\\
&= \int_0^1 (e^z-1)e^{x-z}dx\\
&= (e^z-1)\int_0^1e^{x-z}dx\\
\text{now let $u=x-z$}& \text{ and $du=dx$}\\
&= (e^z-1)\int_{-z}{1-z}^1e^udu\\
&=(e^z-1)(e^u)\Big|_{-z}^{1-z}\\
&=(e-1)e^{-z}(e-1)\\
\end{align*}

Now we solve for $F_z(\tilde{\mu})=\frac{1}{2}$
\begin{align*}
(e-1)e^{-z}(e-1) &= \frac{1}{2}\\
\frac{(e^z-1)}{e^{z}} &= \frac{1}{2(e-1)}\\
e^{-z} &= 1-\frac{1}{2(e-1)}\\
e^z &= \frac{1}{1-\frac{1}{2(e-1)}}\\
z &= -\ln(1-\frac{1}{2(e-1)})\\
\end{align*}

Hence, $\tilde{\mu}\approx 0.34388$.

\item Find $E(X-Y)$ and $\Var(X-Y)$. Find $E(ZX)$.

\newthought{Since} $X$ and $Y$ are independent, we can find $E(X-Y)=E(X)-E(Y)$.

\begin{align*}
E(X) &= \int_0^1xf(x)dx\\
&= \frac{1}{2}\\
\end{align*}
and
\begin{align*}
E(Y)&= \int_0^\infty ye^{-y}dy\\
&=-e^{-y}y\Big|_0^\infty + \int_0^\infty e^{-y}dy\\
&=e^{-y}y\Big|_0^\infty -e^{-y}\Big|_0^{\infty}\\
&= 1\\
\end{align*}

So $E(X-Y)=E(X)-E(Y)= \frac{-1}{2}$

\newthought{The variance} of X-Y can be find in a similiar way.
\[ \Var(X) = E(X^2) - \mu_x^2 \]

So we need find $E(X^2)4$ and $E(Y^2)$.

\begin{align*}
E(X^2) &= \int_0^1x^2f_x(x)\\
&= \frac{x^3}{3}\Big|_0^1\\
&=\frac{1}{3}\\
\end{align*}

We have $\Var(X) = \frac{1}{3} - \frac{1}{4} = \frac{1}{12}$

Next, find $E(Y^2)$
\begin{align*}
E(Y^2) &= \int_0^\infty y^2e^{-y}dy\\
&= -e^yy^2\Big|_0^\infty + 2\int_0^\infty e^{-y}ydy\\
&= -2e^{-x}x|\Big|_0^1 + 2e^{-x}\Big|_0^\infty\\
&=2
\end{align*}

Now we have $\Var(Y) = 2 - 1=2$.  Therefore $\Var(X-Y)=\frac{-11}{12}$

\newthought{Since} $Z=X+Y$,
\begin{align*}
E(ZX)&=E(X^2+XY)\\
&= E(X^2) + E(XY)\\
&= E(X^2) + E(X)E(Y)\\
&= \frac{1}{3} + \frac{1}{2}\\
&= \frac{5}{6}
\end{align*}

\end{enumerate}


\subsection{Question 7}
For $X$ following a standard normal distribution, find $E(X^3)$ and $E(X^4)$.  For $X\sim N(\mu,\sigma^2)$, find $E(X^3)$ and $E(X^4)$ using your answers for the standard normal and no futher calculus.

The PDF of the standard normal is
\[\Phi = \frac{e^{-\frac{1}{2}x^2}}{\sqrt{2\pi}}. \]

Also, note that $E(X)=\mu=0$ and $E(X^2) = 1$ , since $1 = E(X^2) - 0$.
\begin{align*}
E(X^3)&= \int_{-\infty}^{\infty} x^3\frac{e^{-\frac{1}{2}x^2}}{\sqrt{2\pi}}dx\\
&=\frac{1}{\sqrt(2\pi}\int_{-\infty}^{\infty}x^3e^{\frac{-x^2}{2}}dx\\
\text{let $u=x^2$}& \text{ and $du=2xdx$}\\
&=\frac{1}{2\sqrt(2\pi}\int_{-\infty}^{\infty}ue^{\frac{-u}{2}}du\\
\end{align*}
Integrate by parts using $f=u$, $df=du$, and $g=\frac{-1}{e^{\frac{u}{2}}}$
\begin{align*}
E(X^3)&=\frac{1}{2\sqrt{2\pi}}([ue^{\frac{-u}{2}}\Big|_{-\infty}^{\infty}]+\int_{-\infty}^{\infty}2e^{\frac{-u}{2}}du)\\
&=0 + \int_{-\infty}^{\infty}\frac{2e^{-\frac{u}{2}}}{2\sqrt{2\pi}}du\\
&= \int_{-\infty}^{\infty}\frac{e^{-\frac{u}{2}}}{\sqrt{2\pi}}du\\
&= 2\int_{-\infty}^{\infty}x\frac{e^{-\frac{x^2}{2}}}{\sqrt{2\pi}}dx\\
&= 2*E(X)\\
&= 0\\
\end{align*}

Now, we'll find $E(X^4)$.  Let $u=\frac{x^2}{2}$ and $du=xdx$.

\begin{align*}
E(X^4)&=\int_{-\infty}^{\infty} x^4\frac{e^{-\frac{1}{2}x^2}}{\sqrt{2\pi}}dx\\
&=\frac{1}{\sqrt{2\pi}} \int_{-\infty}^{\infty}(2u)^{\frac{3}{2}}e^{-u}du\\
&\text{*since standard normal distribution is symmetric}\\
&=\frac{2}{\sqrt{2\pi}} \int_{0}^{\infty}(2u)^{\frac{3}{2}}e^{-u}du\\
&=\frac{2^{\frac{5}{2}}}{\sqrt{2\pi}}\int_{0}^{\infty}(u)^{\frac{5}{2}-1}e^{-u}du\\
&=\frac{2^{\frac{5}{2}}}{\sqrt{2\pi}}\Gamma(\frac{5}{2})\\
&= \frac{4\sqrt{2}}{\sqrt{2}\sqrt{\pi}} \frac{3\sqrt{\pi}}{4}\\
&= 3
\end{align*}

\newthought{For} $X\sim(\mu,\sigma^2)$, $E(X^3)=0$ and $E(X^4)=3\sigma^4$.  Admittedly, I'm not sure how or why from my answers above.  I imagine that since standard deviation will be a constant it will just fall out of the intergrals...


\subsection{Question 8}
Consider a Poisson process with paremter $\lambda$.  Let $X$ be the number of events in $(0,t_2]$ and $Y$ the number of events in $(t_1,t_3]$ for $0<t_1<t_2<t_3$ so that the intervals are guaranteed to overlap.

\begin{enumerate}
\item Find the mean and variance of $X-Y$.

\newthought{Consider first} the expected values of $X$ and $Y$, independently.

\begin{align*}
E(X)&= \sum_{k=0}^\infty \frac{k\lambda(t_2)^k}{k!}e^{-\lambda(t_2)}\\
&= [\lambda(t_2)]e^{-\lambda(t_2)} \sum_{k=1}^\infty\frac{[\lambda(t_2)]^{k-1}}{(k-1)!}\\
&=[\lambda(t_2)]e^{-\lambda(t_2)} \sum_{j=1}^\infty\frac{[\lambda(t_2)]^{j}}{(k-1)!}\\
&=[\lambda(t_2)]e^{-\lambda(t_2)}e^{\lambda t_2}\\
&= \lambda(t_2)
\end{align*}

For $E(Y)$ we can following the same process, resuling in

\begin{align*}
E(Y) &= \lambda(t_3-t_1).\\
\end{align*}

So
\begin{align*}E(Y)-E(X)=\lambda(t_3-t_1)-\lambda(t_2)
\end{align*}

The variance is defined.
\[ \Var(Y-X) = \Var(Y)-\Var(X)-2\Cov(X,Y) \]
We can use the the moment generating function for the Poisson distribution.
\begin{align*}
M(t) &= \sum_{k=0}^\infty e^{tk}\frac{\lambda^k}{k!}e^{-\lambda}\\
&= \sum_{k=0}^\infty \frac{(\lambda e^t)^k}{k!}e^{-\lambda}\\
&= e^{-\lambda}e^{\lambda(e^{t}-1)}\\
&= e^{\lambda(e^t-1)}\\
\end{align*}

Now differentiate
\begin{align*}
M'(t)&=\lambda e^te^{\lambda(e^t-1)}\\
M''(t)&= \lambda e^te^{\lambda(e^t-1)} + \lambda^2e^{2t}e^{\lambda(e^t-1)}\\
\end{align*}

We already know the value at the first moment $E(X)$ so just evualate the second derivative at $t=0$.

\[ E(X^2)= (\lambda t_2)^2 + \lambda t_2 \]

Therefore we have

\[ \Var(X) = E(X^2)- [E(X)]^2 = \lambda t_2. \]

Similarily

\[ \Var(Y) = E(Y^2)-[E(X)]^2 = \lambda(t_3-t_1). \]

It remains to find $\Cov(XY)$.
\[ \Cov(X,Y) = E(XY) - E(X)E(Y) \]

The last item to find is $E(XY)$.  The joint PMF
\[ p_k= \frac{[\lambda(t_2-t_1)]^ke^{-[\lambda(t_2-t_1)]}}{k!} \]

Using the same methods for find the $E(X)$ and $E(Y)$

\[ E(XY) = \lambda(t_2-t_1) \]

Plugging values in
\begin{align*}
\Cov(X,Y) = \lambda(t_2-t_1) - \lambda(t_3-t_1)\lambda(t_2)
\end{align*}

So
\begin{align*}
\Var(Y-X) &= \Var(Y) - \Var(X) - 2\Cov(X,Y)\\
&= \lambda(t_3-t_1) - \lambda t_2 - 2[\lambda(t_2-t_1) - \lambda^2(t_3-t_1)(t_2)]\\
\end{align*}

\item Find $E(Y \mid X)$. Verify that $E[E(Y\mid X)]$ equals $E(Y)$.

\newthought{In general} $P(Y\mid X) = \frac{P(X,Y)}{P(Y)}$.  In this case, if $p$ is the PMF of the Poisson distribution.
\[p_{Y\mid X} = \frac{p_{XY}}{p_Y} \]

\begin{align*}
p_{XY} &= \frac{(t_2\lambda)^xe^{-(t_2)\lambda}}{x!}\frac{[(t_3-t_1)\lambda]^{y-x}e^{-(t_3-t_1)\lambda}}{(y-x)!}\\
\end{align*}

The conditional expection of $Y$ given $X$ is
\[ E(Y \mid X) = \sum_y yp_{Y\mid }(y\mid x) \]

\begin{align*}
p_{Y\mid X}(y\mid n)&=\frac{(t_2\lambda)^xe^{-(t_2)\lambda}}{x!}\frac{[(t_3-t_1)\lambda]^{y-x}e^{-(t_3-t_1)\lambda}}{(y-x)!}\\
&*\frac{1}{ \frac{[(t_3-t_1)\lambda]^{y-x}e^{-(t_3-t_1)\lambda}}{(y!)}}\\
&= \frac{x!}{x!(y-x!)}t_2^x(t_3-t_1)^{y-x}\\
\end{align*}

\newthought{The last} thing to do is verifty $E[Y\mid X = E(Y)$.

\begin{align*}
E[E(Y\mid X)]&= \sum_y yP_{Y\mid X}(y\mid x)\\
&= \sum_y y \frac{x!}{x!(y-x!)}t_2^x(t_3-t_1)^{y-x}\\
E[E(Y\mid X)] \sum_xp_X(x) &= \sum_y y \sum_x \frac{x!}{x!(y-x!)}t_2^x(t_3-t_1)^{y-x}p_X(x)\\
\end{align*}
The law of total probability states
\[ p_Y(y) = \sum_xp_{Y\mid X}(y \mid x) p_{X}(x) \]
So
\begin{align*}
\sum_y y \sum_x \frac{x!}{x!(y-x!)}t_2^x(t_3-t_1)^{y-x}p_X(x)&=\sum_y p_y(Y)\\
&= E(Y)\\
\end{align*}
\end{enumerate}
\end{document}
\grid
